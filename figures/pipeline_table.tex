\centering
    \resizebox{\columnwidth}{!}{%
\begin{tabular}{l | c c c | c c c}
\hline
&\multicolumn{3}{c|}{\textbf{Dev Set Scores}} & \multicolumn{3}{c}{\textbf{Test Set Scores}}  \\
\textbf{Pipeline Name} & \textbf{Q only} & \textbf{Q+A} & \textbf{\averitec{}} & \textbf{Q only} & \textbf{Q+A} & \textbf{AVeriTeC} \\ \hline
\textbf{GPT-4o (full-featured pipeline)}      & \textbf{0.46} & \textbf{0.29} & \textbf{0.42} & 0.46 & \textbf{0.32} & \textbf{0.50}\\
GPT-4o (simplified)         & 0.45 & 0.28 & 0.38 & 0.45 & 0.30 & 0.47 \\
Claude-3.5 (full-featured)             & 0.43 & 0.28 & 0.35 & 0.42 & 0.30 & 0.46 \\
GPT-4o (with DeBERTa classifier)              & 0.45 & 0.28 & 0.36 & -- & -- & --\\
\averitec{} aseline            & 0.24 & 0.19 & 0.09 & 0.24 & 0.20 & 0.11\\
\hline
Llama 3.1 70B (full-featured) & \textbf{0.46} & 0.27 & 0.36 & \textbf{0.47} & 0.29 & 0.42\\
\bottomrule
\end{tabular}
}
\caption{Comparison of Pipeline Scores on Dev and Test Sets. \review{Q, Q+A are Hu-METEOR scores against gold data,} AVeriTeC scores \review{are calculated as referred in section~\ref{avscore} thresholded at 0.25}. \say{Full-featured} pipelines use the all the improvement techniques introduced in section~\ref{sec:system}, while the simplified pipeline omits the dynamic few-shot learning, answer-type-tuning and Likert-scale confidence emulation described in section~\ref{sec:generation}}
\label{tab:pipeline_scores}