% arara: pdflatex: { synctex: yes }
% arara: makeindex: { style: ctuthesis }
% arara: bibtex

% The class takes all the key=value arguments that \ctusetup does,
% and a couple more: draft and oneside
\documentclass[oneside]{ctuthesis}

\ctusetup{
%	preprint = \ctuverlog,
	mainlanguage = english,
	otherlanguages = {czech},
	title-english = {NLP Methods for Automated Fact-Checking},
	xdoctype = M,
	xfaculty = F3,
	department-english = {Department of Computer Science},
	author = {Ing. Herbert Ullrich},
	supervisor = {Ing. Jan Drchal, Ph.D.},
	fieldofstudy-english = {Informatics},
	subfieldofstudy-english = {Natural Language Processing},
	keywords-czech = {Fact-checking, Natural Language Inference, Claim Generation, Transformers, LLMs},
	keywords-english = {Fact-checking, Natural Language Inference, Claim Generation, Transformers, LLMs},
	day = 21,
	month = 8,
	year = 2023,
	pkg-listings = true
%	monochrome = true,
%	layout-short = true,
}

\ctuprocess

\ctutemplateset{maketitle twocolumn default}{
	\begin{twocolumnfrontmatterpage}
		%\ctutemplate{twocolumn.thanks}
		%\ctutemplate{twocolumn.declaration}
		%\ctutemplate{twocolumn.abstract.in.titlelanguage}
		%\ctutemplate{twocolumn.abstract.in.secondlanguage}
		\ctutemplate{twocolumn.tableofcontents}
		\ctutemplate{twocolumn.listoffigures}
	\end{twocolumnfrontmatterpage}
}

% Theorem declarations, this is the reasonable default, anybody can do what they wish.
% If you prefer theorems in italics rather than slanted, use \theoremstyle{plainit}
\theoremstyle{plain}
\newtheorem{theorem}{Theorem}[chapter]
\newtheorem{corollary}[theorem]{Corollary}
\newtheorem{lemma}[theorem]{Lemma}
\newtheorem{proposition}[theorem]{Proposition}

\theoremstyle{definition}
\newtheorem{definition}[theorem]{Definition}
\newtheorem{example}[theorem]{Example}
\newtheorem{conjecture}[theorem]{Conjecture}

\theoremstyle{note}
\newtheorem*{remark*}{Remark}
\newtheorem{remark}[theorem]{Remark}

\DeclareMathOperator*{\argmin}{arg\!min}
\DeclareMathOperator*{\argmax}{arg\!max}


% Only for testing purposes
\listfiles
\usepackage[pagewise]{lineno}
\usepackage{lipsum,blindtext}
\usepackage{mathrsfs} % provides \mathscr used in the ridiculous examples

% TODO: filter out unnecessary pckgs
%\usepackage[breaklinks=true]{hyperref}
%\usepackage{breakcites}
\usepackage{cite}
%\usepackage{mathtools}
%\usepackage{amsmath}
%\usepackage{amsfonts}
%\usepackage{amssymb}
%\usepackage{algpseudocode}
%\usepackage{algorithm}
\usepackage[ruled,lined,linesnumbered, commentsnumbered]{algorithm2e}
%\usepackage{listings}
%\usepackage{caption}
\usepackage{subcaption}
\usepackage{enumerate}
\usepackage{tabularx}
%\usepackage{hyperref}
\usepackage{tablefootnote}

\def\"#1{``#1''}
\def\btn#1{{\itembox{\textbf{\textsf{#1}}}}}
\def\db#1{{{\textit{\textsf{#1}}}}}
\def\tnula{\hyperref[t0]{$\textsf{T}_{\textsf{0}}$}}
\def\tjednaa{\hyperref[t1a]{$\textsf{T}_{\textsf{1a}}$}}
\def\tjednab{\hyperref[t1b]{$\textsf{T}_{\textsf{1b}}$}}
\def\tdvaa{\hyperref[t2a]{$\textsf{T}_{\textsf{2a}}$}}
\def\tdvab{\hyperref[t2b]{$\textsf{T}_{\textsf{2b}}$}}
\def\tdva{\hyperref[t2a]{$\textsf{T}_{\textsf{2}}$}}

\begin{document}

\maketitle

%!TEX ROOT=../ctutest.tex

\chapter{Introduction}
\label{chap:intro}

Our dissertation, as well as our long-term research, centers around the field of \textit{automated fact checking} through the means of Natural Language Processing and its modern methods.
The work consists of the analysis of the whole fact-checking process, its subdivision and simplification into tasks that can be efficiently addressed using the current state-of-the-art NLP methods, collection of data appropriate to benchmark such tasks, delivery of example solutions and their validation against similar research in other languages and related tasks.

Our main focus are the fact-checking-related tasks in the West Slavic languages (Czech, Slovak and Polish) and secondarily in English.
Our contribution has so far been the collection and publication of novel datasets for the fact-checking task and its subroutines, models trained for the tasks and their debate, including the ongoing establishment of metrics that would rate the model success and error rates in terms close to the human notion of \textit{facticity} (which proves to be a challenge on its own, requiring another round of novel research). 

Our doctoral aim is to cover every step on the path from gathering a factual claim -- for example, extracting it from a political debate -- to predicting its veracity verdict and justifying it rigorously with hard data.
With the recent boom in NLP beginning with the advent of transformer networks and later the Large Language Models, prompting and few-shot learning, a significant part of the research is and has to be an appropriate and timely adoption of new ever-evolving sota NLP solutions, based on well-designed studies in our specific context.

Overall, our agenda is to follow up on our published research on fact checking in Czech with methods that reiterate on our results in other languages and evolving our previous methodology based on transformer \textit{pre-training \& fine-tuning} paradigm to a computationally feasible design based on LLMs. 
We want to establish the task of \textit{claim generation} among the other commonly benchmarked NLP tasks within the scientific community, adjacent to that of \textit{abstractive summarization}.
We aim to give safeguards and explanations to the model decisions with human-understandable metrics, in particular revealing hallucinations -- a common problem of modern day LLMs.

The goal of this study is to show the directions we are taking to address these challenges, reasoning behind them, our research questions and current results that motivated them.

\section{Motivation}
\label{sec:motivation}

The spread of misinformation in the online space has a growing influence on the Czech public~\cite{stem}. It has been shown to influence people's behaviour on the social networks~\cite{Lazer1094} as well as their decisions in elections~\cite{10.1257/jep.31.2.211}, and real-world reasoning, which has shown increasingly harmful during the COVID-19 pandemic~\cite{BARUA2020100119}.

The recent advances in artificial intelligence and its related fields, in particular the recommendation algorithms, have contributed to the spread of misinformation on social media~\cite{doi:10.1177/2056305119888654}, as well as they hold a large potential for automation of the false content generation and extraction of sensational attention-drawing headlines -- the \"{clickbait} generation~\cite{shukai}.

Recent research has shown promising results~\cite{fever2} in false claim detection for data in English, using a trusted knowledge base of true claims (for research purposes typically fixed to the corpus of \textsf{Wikipedia} articles), mimicking the \textit{fact-checking} efforts in journalism.

Fact-checking is a rigorous process of matching every information within a \textit{factic claim} to its \textit{evidence} (or \textit{disproof}) in trusted data sources to infer the claim veracity and verifiability. In exchange, if the trusted \textit{knowledge base} contains a set of \"{ground truths} sufficient to fully infer the original claim or its negation, the claim is labelled as {\techbf{supported}} or {\techbf{refuted}}, respectively. If no such \textit{evidence set} can be found, the claim is marked as {\techbf{unverifiable}}\footnote{Hereinafter labelled as \texttt{NOT ENOUGH INFO}, in accordance to related research.}.


\section{Challenges}

\begin{figure}
    \includegraphics[width=14cm]{fig/framework.pdf}
    \caption{Automated fact-checking pipeline, reprinted from~\cite{guo-etal-2022-survey}}
    \label{fig:framework}
\end{figure}

Despite the existence of end-to-end fact-checking services, such as \url{politifact.org} or \url{demagog.cz}, the human-powered approach shows weaknesses in its scalability. By design, the process of finding an exhaustive set of evidence that decides the claim veracity is much slower than that of generating false or misguiding claims. Therefore, efforts have been made to move part of the load to a computer program that can run without supervision.

The common research goal is a fact verification tool that would, given a claim, semantically search provided knowledge base (stored for example as a \textit{corpus} of some natural language), propose a set of evidence (e. g. $k$ semantically nearest paragraphs of the corpus) and suggest the final verdict (Figure \ref{fig:pipeline}). This would reduce the fact-checker's workload to mere adjustments of the proposed result and correction of mistakes on the computer side. 

The goal of the ongoing efforts of {\textsf{FactCheck}} team at {\textsf{AIC CTU}}, as addressed in the works of~\cite{rypar,dedkova} and~\cite{gazo} is to explore the state-of-the-art methods used for fact verification in other languages, and propose a strong baseline system for such a task in Czech.


\section{A word on the Transformers}
\label{sec:transformers}
For the past six years, the state-of-the-art solution for nearly every Natural Language Processing task is based on the concept of \textit{transformer networks} or, simply, \textit{Transformers}. This has been a major breakthrough in the field by~\cite{vaswani}, giving birth to the famous models such as \textsf{Google}'s \textsf{BERT}~\cite{bert} and its descendants, or the \textsf{OpenAI}'s \textsf{GPT-3}~\cite{gpt3}.

In our proposed methods, we use Transformers in every step of the fact verification pipeline. Therefore, we would like to introduce this concept to our reader to begin with. 

Transformer is a neural model for \textit{sequence-to-sequence} tasks, which, similarly e.g. to the \textit{LSTM-Networks}~\cite{lstm}, uses the Encoder--Decoder architecture. Its main point is that of using solely the \textit{self-attention} mechanism to represent its input and output, instead of any sequence-aligned recurrence~\cite{vaswani}.

In essence, the \textit{self-attention} (also known as the \textit{intra-attention}) transforms every input vector to a weighted sum of the vectors in its neighbourhood, weighted by their \textit{relatedness} to the input. One could illustrate this on the \textit{euphony} in music, where every tone of a song relates to all of the precedent ones, to some more than to the others.

The full Transformer architecture is depicted in Figure~\ref{fig:transformer}.
%--- FIG: UTF forms
\begin{figure}
\includegraphics[width=9cm]{fig/transformer.pdf}
\caption{Transformer model architecture, reprinted from~\cite{vaswani}}
\label{fig:transformer}
\end{figure}
%--- /FIG



\section{Dissertation minimum study outline}
 
\begin{itemize}
\item {\techbf{Chapter~\ref{chap:intro}}} introduces the dissertation topic, motivates the research sets up our challenges for the future research 

\item {\techbf{Chapter~\ref{chap:sota}}} examines the most relevant research in the field and tries to highlight the recent paradigm shift from models trained for a single task to a single large models that perform well in everything

\item {\techbf{Chapter~\ref{chap:contribution}}} explains our current contributions to the field of automated fact-checking and NLP in Chech

\item {\techbf{Chapter~\ref{chap:plan}}} describes our plan for the dissertation and justifies the directions we are taking

\item Finally, {\techbf{Chapter~\ref{chap:conclusion}}} concludes the study with a wrapup of its findings

\end{itemize}


%!TEX ROOT=../ctutest.tex

\chapter{State of the Art}
This chapter will first describe the originally popular models for NLP such as BERT and its relatives and then the paradigm shift from pre-train+fine-tune frameworks to LLMs (possibly few-shot learned or adapted with LoRA). 

\todo{Citations: Lora, daniil, Bert paper, GPT4 debilní report}
\label{chap:sota}
\section{Pre-train + Fine-tune}
\label{sec:pretrain}
%!TEX ROOT=../ctutest.tex

\chapter{Current Contribution}
\label{chap:contribution}

\textit{We have collected novel data for the fact-checking task in our application context, emulated and scraped inavailable datasets making them public or readying them for doing so, we have established numerous state-of-the-art models and we are currently working on establishing the topic of claim generation as a summarization-related NLP task.}

\section{Datasets}
Having the automated fact-checking scheme established in chapter~\ref{chap:sota}, every machine-learning solution must start with the choice or collection of appropriate training data.
Due to the novelty of the task in Czech and other West Slavic languages, I explored a multitude of ways to acquire such data, many of them resulting in a publicly available dataset in our Huggingface repository~\footnote{\url{https://huggingface.co/ctu-aic}}, beginning to be reused by others. 

\subsection{\FCZ}
An early \"{temporary benchmark} for our endeavours in adapting the FEVER~\cite{fever} task for the Czech context was the \FCZ~\cite{lrev} dataset.

In~\cite{diplomka}, I have proposed a simple FEVER data transduction scheme that can be simplified as follows:

\begin{enumerate}
    \item Each FEVER claim is translated using the (at the time maturing) Machine Translator
    \item Evidence from English Wikipedia is not translated using MT, but mapped onto its Czech-Wikipedia counterpart using the publicly available Wikidata\footnote{Used, for example, for showing the \"{see this article in other languages} suggestions in Wikipedia sidebar}
    \item Data with any loss in evidence due to the step 2. is discarded
\end{enumerate}

This design was relatively cheap to compute (as translating the whole 2017 Wikipedia corpus would have been a long and wasteful computation), delivering an open-license dataset of 127K claims, their labels and evidence justifications. My hope was, as both the 2017 EnWiki and our 2020 CsWiki corpus only featured the first paragraph (abstract) of each article, a document-level alignment could be assumed -- both the Czech and English text always summarize the basic facts about the same entity.

This showed to be only partly true as a later human annotation on a 1\% sample of \FCZ data showed that about a third of data exhibits some levels of noise, mostly introduced during dataset translation~\cite{lrev}.

While noisy, the \FCZ data still got its use in training of the information retrieval schemes of~\cite{rypar,gazo,lrev} used to this day and is openly available\footnote{\url{https://huggingface.co/datasets/ctu-aic/csfever}} under a CC license.

My research on it also motivated a creation of a inference-only version of the dataset, which does not support the Information Retrieval task and therefore, does not require the mapping of evidence into a live version of Wikipedia.
Therefore, only the EnWiki \textit{excerpts} needed to build evidence can be translated, bringing down the computational difficulty and enabling me to deliver a dataset without the transduction noise called \FCZNLI\footnote{\url{https://huggingface.co/datasets/ctu-aic/csfever_nli}}. 

Another round of research \FCZ motivated and I supervised was the successful thesis of~\cite{mlynar}, modernizing the data and machine-translation methods into the 2023 state of the art.
\cite{mlynar} further experimented with methods of automated noise detection and removal, which has not shown to be an efficient way to tackle the issue of high noise in \FCZ.

However, it delivers a partly cleaned versions of it\footnote{\url{https://huggingface.co/datasets/ctu-aic/csfever_v2}} and motivates a future research of generating such data differently, using a claim generation scheme like that from~\cite{pan2021zeroshot}.

\label{sec:datasets}
\begin{figure}
    \makebox[\textwidth][c]{
    \includegraphics[height=8cm]{fig/fcheck/claim_extraction.pdf}
    \includegraphics[height=8cm]{fig/fcheck/mutation.pdf}
    \includegraphics[height=8cm]{fig/fcheck/annotation.pdf}
    }
    \caption{{\techbf FCheck} -- an open-source platform for fact-checking dataset collection I developed for TAČR project; collects data for claim generation, information retrieval and natural language inference tasks}
    \label{fig:fcheck}
\end{figure}
\subsection{FCheck Annotations Platform}
The imperfections in translated \FCZ data, as well as the ongoing colaboration with ČTK and the Faculty of Social Sciences brought me to also look for ways how to hand-annotate a whole new natively Czech dataset, which would both lack the noise of translated data and take the task of automated fact checking to next level, replacing a rigid, simple Wikipedic data with a more \"{real world} news report corpus of ČTK.

Figure~\ref{fig:fcheck} shows 
\subsection{\CTK}
\subsection{Other NLP datasets in West Slavic languages}
\begin{enumerate}
    \item {\techbf Translated NLI datasets} -- SNLI, ANLI, MultiNLI, 
    \item SmeSum, CTKSum, CsFEVERSum
    \item Polish summarization data
\end{enumerate}
\section{Models}
\label{sec:models}
\section{Publications}
\label{sec:publications}
\section{Applications}
\label{sec:applications}

\begin{figure}
    \includegraphics[width=16cm]{fig/cedmo.pdf}
    \caption{Factual claim extraction application done for the CEDMO project}
    \label{fig:framework}
\end{figure}

Here we will show off the demonstration tools, as well as our open-source platform \url{https://fcheck.fel.cvut.cz} and currently running claim extraction tools. 
%!TEX ROOT=../ctutest.tex

\chapter{Dissertation plan}
\label{chap:plan}

\begin{enumerate}
    \item \textbf{Claim extraction metrics proposal based on factuality of summarization}
    \item \textbf{Claim extraction paradigm that benchmarks best in the newly given metrics}
    \item Systems for NLI built on top of LoRA paradigm to score best in the task, as showed promising by Daniil
    
\end{enumerate}
%!TEX ROOT=../ctutest.tex

\chapter{Conclusion}
\label{chap:conclusion}

%\bibliographystyle{amsalpha}
\bibliographystyle{apalike}
\bibliography{minimum}
\appendix
%\chapter{Czech-English data translations}
\section{Translated figures}

\begin{figure}[H]
\centering
 \fbox{\includegraphics[width=\textwidth]{fig/demagog_en.pdf}}
\caption[English Translation of Figure~\ref{fig:demagog}]{Translated fact verification from Czech portal \textsf{Demagog.cz} -- original in Figure~\ref{fig:demagog}}
\label{trans:demagog}
\end{figure}

\begin{figure}
\thisfloatpagestyle{empty}
\vspace{-2cm}
 \makebox[\textwidth][c]{\includegraphics[width=18cm]{fig/annotation_en.pdf}}
\caption[English Translation of Figure~\ref{fig:annotation}]{The labelling interface of \textsf{FCheck} platform. Czech original in Figure~\ref{fig:annotation}}
\label{trans:annotation}
\end{figure}

\chapter{Acronyms}
\begin{description}
\item[BERT] Bidirectional Encoder Representations from Transformers
\item[GPT] Generative Pre-trained Transformer 
\item[FEVER] Fact Extraction and Verification -- series of Shared tasks focused on fact-checking
\item[CLI] Command-Line Interface
\item[NLI] Natural Language Inference
\item[ČTK] Czech Press Agency
\end{description}


\end{document}