%!TEX ROOT=../dissertation.tex

\chapter{Conclusion}
\label{chap:conclusion}
\label{chap:proposed}
Our work has addressed the lack of a fact-verification dataset in Czech in two ways:

Firstly, it established a scheme of transferring an English \textsf{ENWiki}-corpus-based \textsf{FEVER} dataset to the Czech language using Machine Translation and the cross-lingual mapping of \textsf{WikiMedia API}, obtaining a set of a total of 127K translated claims along with their veracity labels and evidence within the \textsf{CSWiki} corpus, which we call the \textsf{FEVER CS} dataset.

Secondly, we prepared a series of human-annotation experiments, that were conducted with 163 annotators, utilizing the collaboration with the \textsf{Faculty of Social Sciences of Charles University} towards collecting about 10K \db{Claim} and \db{Label} data points stemming from our application-specific \textsf{ČTK Archive} knowledge base, achieving an inter-annotator aggreement of 0.63, measured using the 4-way Fleiss' $\kappa$.

For the annotations, we have built a novel annotation platform from the ground up, naming it the \textsf{FCheck Annotations Platform} and publishing it as an open-source project. Subsequently, we have used our platform to export the novel \textsf{ČTK} dataset, that contains 3,295 textual claims along with their veracity labeling and \textsf{ČTK}-based sets of conclusive evidence, extracted from the results earlier annotation experiments.

Finally, we deem this dataset eligible for training statistical models for the task of Natural Language Inference, demonstrating the usage of our data on transfer-learning a triple of Transformer networks -- \textsf{XLM-RoBERTa}, \textsf{SlavicBERT} and \textsf{multilingualBERT}, the first of which scores \textbf{85.5\%} micro-$F_1$ on the \textsf{ČTK} claim veracity labeling task.

\section{Proposed solutions}
At the end of every chapter, we provide a textual \textit{wrapup} of its result, along with remarks on its reproducibility, and, where possible, a ready-made solution in form of an open-source code, or prebuilt models and datasets shared through a public cloud-storage link.

This is to encourage any future research on the topic, as well as to challenge our results and their credibility.

\section{Future research goals}
Our work at \textsf{AIC FactCheck} is far from over. After the publication of the \textsf{ČTK} dataset and the baseline NLI classifier, we are about to pursue some of the following goals that arised from the findings in previous chapters:

\begin{enumerate}
    \item The solution for the \textit{overfitting} issue from Chapter 7 should be examined, using some of the attached localized \textsf{SNLI},  \textsf{MultiNLI},  \textsf{ANLI}  and \textsf{FEVER-NLI} sets, as outlined in the previous Chapter \textit{wrapup}
    \item The novel monolingual Czech \textbf{\textsf{CZERT}} model is to be trained and examined on the same tasks as the other models from the Chapter 7
    \item \textbf{\textsf{The FEVER CS Baseline}} end-to-end containerized pipeline should be updated with our resulting models and that of~\cite{rypar} for the production purposes
    \item The set of \tdvab{} \textbf{\textsf{Claim Mutations}} (Figure~\ref{fig:mutations}) collected by the \textsf{FCheck} platform is to be examined and challenged with the dataset balance in mind, as the same set of mutation tasks yielded a significantly label-unbalanced dataset to \textit{us} (Chapter~\ref{chap:dataset}), to~\cite{fever} and to~\cite{danish}, all of them in favour of the \texttt{SUPPORTS annotation}
\end{enumerate}
